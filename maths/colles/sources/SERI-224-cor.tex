Déjà, $S_k$ est bien définie, car c'est une série de Riemann avec $k>1$.\\

De plus, pour tout $N\in\N^*$, $\displaystyle\sum_{n=1}^{2N+1} \dfrac{1}{ n^k}=\displaystyle\sum_{n=1}^{N} \dfrac{1}{(2 n)^k}
+\displaystyle\sum_{n=0}^{N} \dfrac{1}{(2 n+1)^k}$. Par passage à la limite concernant trois séries à termes réels positifs quand $N\longrightarrow\pinf$, les réels $T_k$ et $V_k=\displaystyle\sum_{n=1}^{+\infty} \dfrac{1}{(2 n)^k}$ sont bien définis et $S_{k}=T_k+V_k$.\\
De plus $V_k=\dfrac{1}{2^k}\displaystyle\sum_{n=1}^{\pinf} \dfrac{1}{ n^k}=\dfrac{1}{2^k}S_k$.\\
Ainsi $S_k=\dfrac1{1-\frac1{2^k}} T_k$.
