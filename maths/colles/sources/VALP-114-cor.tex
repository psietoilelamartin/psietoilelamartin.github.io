$\rg J=1$ donc $\dim\Ker J=n-1$. On remarque que $v_1=(1,-1,0,\cdots,0)$, $v_2=(0,1,-1,0,\cdots,0)$, $v_{n-1}=(0,\cdots,0,1,-1)$ sont dans le noyau de $J$, et ils forment une famille libre car échelonnée. Ainsi, par raison de cardinal,  ils forment une base de $\Ker J$.\\
Enfin, si $v_n=(1,\cdots,1)$, alors $Jv_n=nv_n$, donc c'est un \vecp\ de $J$ pour la \valp\ 1. Ainsi $\dim E_n(J)\geq 1$.
Mais les \seps\ sont en somme directe et $\dim E_0(J)+\dim E_n(J)\geq n$. Donc nécessairement $\dim E_n(J)=1$, et il ne peut y avoir d'autre \valp.\\
Finalement, $J$ a deux \valps\ : $0$ et $n$. Et $E_0(J)=\Vect(v_1,\ldots,v_{n-1})$, et $E_n(J)=\Vect v_n$.



% Dans $\cB=(v_1,\cdots,v_n)$, $\MAT_{\cB}(J)=\bpm \scalebox{1.5}{0}_{n-1}&0\\0&1\epm$.
