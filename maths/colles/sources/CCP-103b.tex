% CCP 2023 exo 103
% \textbf{Remarque}: les questions 1. et 2. sont indépendantes.\\

Soit $(\Omega,\mcal{A},P)$ un espace probabilisé.
\begin{enumerate}
\item
% \begin{enumerate}
% \item
 Soit $\left( \lambda_1,\lambda_2 \right) \in \left(\left] 0,+\infty \right[\right) ^2$.\\
Soit $X_1$ et $X_2$ deux variables aléatoires définies sur $(\Omega,\mcal{A},P)$.\\
On suppose que  $X_1$ et $X_2$ sont indépendantes et suivent des lois de Poisson, de paramètres respectifs  $\lambda_1$ et $\lambda_2$.\\
Déterminer la loi de $X_1+X_2$.
% \item
% En déduire l'espérance et la variance de $X_1+X_2$.
% \end{enumerate}
\item
Soit $p\in\left]  0,1\right]$. Soit $\lambda \in \left] 0,+\infty \right[$. \\
Soit $X$ et $Y$ deux variables aléatoires définies sur $(\Omega,\mcal{A},P)$. \\
On suppose que  $Y$ suit une loi de Poisson de paramètre $\lambda$.\\
On suppose que $X(\Omega)=\mathbb{N}$ et que, pour tout $m\in\mathbb{N}$, la loi conditionnelle de $X$ sachant $(Y=m)$ est une loi binomiale de paramètre $(m,p)$.\\
Déterminer la loi de $X$.
\end{enumerate}
