\begin{enumerate}
\item Si $f^p(x)=0_E, f\left(f^p(x)\right)=0_E$, i.e. $f^{p+1}(x)=0_E$. On a donc
$$
x \in F_p \Longrightarrow x \in F_{p+1}
$$

Ou encore $F_p \subset F_{p+1}$.
De plus, si $y \in G_{p+1}$, il existe $x$ tel que $y=f^{p+1}(x)$. Mais alors $y=f^p(f(x))$, donc $y \in G_p$. Et finalement $G_{p+1} \subset G_p$.
\item On est en dimension finie : la suite des dimensions des $F_p$, croissante et majorée, converge vers $\ell$. Mais c'est une suite d'entiers naturels. Elle est donc stationnaire et $\ell\in\N$. Il existe donc $l$ tel que pour tout $k\geq l$, $F_k=F_{k+1}=F_l$ (si un sev est inclus dans un autre et s'ils ont même dimension, ils sont égaux).
On peut alors poser $r=\min\set{l\in\N\ ,\ F_l=F_{l+1}}$. C'est un ensemble d'entiers naturels, non vide d'après le point précédent, donc $r$ existe bien.\\
On montre ensuite par récurrence sur $p$ que :
      \[ \forall p\in\N,~H_p:\textrm{\og} \Ker(f^r) =\Ker(f^{r+p})\textrm{\fg}.
    \].
    \begin{description}
      \item[Initialisation :] $H_0$ est évidente.
      \item[Hérédité :] Soit $p\in \N$, montrons $H_p\Rightarrow H_{p+1}$ et supposons pour cela $H_p$. On a d'abord, par croissance de la suite de sous-espaces vectoriels de $E$ $(\Ker f^n)$, que $ \Ker(f^r) \subset \Ker(f^{r+p+1})$. Ensuite, soit $x\in\Ker(f^{r+p+1})$. On a alors $f^{p+r+1}(x) = f^{p+r}(f(x)) = 0_E$ et donc $f(x) \in \Ker(f^{r+p})$. Par hypothèse de récurrence, on a donc $f(x) \in \Ker f^r$, et donc $f^{r}(f(x)) = f^{r+1}(x) = 0_E$, soit $x\in\Ker f^{r+1}$. D'après la question précédente, on a finalement que $x\in \Ker f^r$ et donc $\Ker(f^{r+p+1}) \subset \Ker f^r$, soit $\Ker(f^{r+p+1}) = \Ker f^r$.
      \item[Conclusion :] Par récurrence, pour tout entier naturel $p$, $H_p$ est vraie.
    \end{description}


% C'est même toujours vrai à partir d'un certain rang. Mais peu importe, il suffit pour l'instant de considérer
% $$
% r=\min \left(\left\{p ; F_p=F_{p+1}\right\}\right)
% $$
%
% Montrons maintenant que
% $$
% F_p=F_{p+1} \Longrightarrow F_{p+1}=F_{p+2}
% $$
%
% Supposons, donc, $F_p=F_{p+1}$. On sait déjà que $F_{p+1} \subset F_{p+2}$, il s'agit donc de montrer l'inclusion inverse. Considérons $x \in F_{p+2}$. Comme on a envie d'utiliser l'hypothèse qui est $F_{p+1} \subset F_p$, il est naturel de chercher à faire apparaître un élément de $F_{p+1}$. Or le fait que $f^{p+2}(x)=0_E$ s'écrit aussi bien $f^{p+1}(f(x))=0_E$, donc $f(x) \in F_{p+1}$. De cela on déduit que $f(x) \in F_p$, c'est-à-dire que $x \in F_{p+1}$. On a donc bien montré ce qu'on voulait et, par récurrence, le résultat demandé s'ensuit.
\item On peut faire le même genre de raisonnement qu'à la question précédente, mais il est plus simple de se souvenir du théorème du rang. En effet, comme pour tout $p$ on a $G_{p+1} \subset G_p$, on a
$$
G_p=G_{p+1} \Longleftrightarrow \operatorname{dim}\left(G_p\right)=\operatorname{dim}\left(G_{p+1}\right)
$$

Mais du théorème du rang on déduit facilement que
$$
\left(\operatorname{dim}\left(G_p\right)=\operatorname{dim}\left(G_{p+1}\right)\right) \Longleftrightarrow\left(\operatorname{dim}\left(F_p\right)=\operatorname{dim}\left(F_{p+1}\right)\right)
$$
et on est ramené à utiliser les résultats de la question précédente. On trouve $r=s$.
\item Comme $r=s$, le théorème du rang fait qu'il nous suffit de montrer que
$$
F_r \cap G_r=\left\{0_E\right\}
$$

Mais si $x \in F_r \cap G_r$, soit $y$ tel que $x=f^r(y)$; de $f^r(x)=0_E$ on déduit que $f^{2 r}(y)=0_E$. Donc $y \in F_{2 r}$. Mais $F_{2 r}=F_r$ d'après 2. Donc $y \in F_r$, donc $x=0_E$, ce qui conclut.
\end{enumerate}
