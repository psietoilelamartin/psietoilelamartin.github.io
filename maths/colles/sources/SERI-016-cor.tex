\begin{enumerate}

\item

Soit $\varepsilon>0$. Comme $u_n\tend 0$, il existe $N\in\mathbb{N}$ tel que pour tout $k\ge N$, $|u_k|\le \varepsilon$.
Pour $n\ge N$,
\[
  |v_n|
  = \left|\frac{1}{n+1}\sum_{k=0}^n u_k\right|
  \le \frac{1}{n+1}\left(\sum_{k=0}^{N-1}|u_k| + \sum_{k=N}^n |u_k|\right)
  \le \frac{A}{n+1} + \frac{n-N+1}{n+1}\,\varepsilon,
\]
où $A=\dsum_{k=0}^{N-1}|u_k|$. Choisissons $n$ assez grand pour avoir $\dfrac{A}{n+1}\le \varepsilon$ (par exemple $n+1\ge A/\varepsilon$). Alors
\[
  |v_n| \le \varepsilon + \varepsilon = 2\varepsilon.
\]
Comme $\varepsilon>0$ est arbitraire, on en déduit $v_n\tend 0$.

\item \textbf{(Théorème de Césaro) Si $u_n$ converge, alors $v_n$ converge et $\displaystyle \lim v_n=\lim u_n$.}

Supposons $u_n\tend \ell\in\mathbb{C}$. Posons $w_n= u_n-\ell$, alors $w_n\tend 0$.
On écrit
\[
  v_n
  = \frac{1}{n+1}\sum_{k=0}^n u_k
  = \frac{1}{n+1}\sum_{k=0}^n (\ell + w_k)
  = \ell + \frac{1}{n+1}\sum_{k=0}^n w_k.
\]
Grâce à la première question, comme $w_k\tend 0$, on a $\dfrac{1}{n+1}\dsum_{k=0}^n w_k \to 0$. Donc $v_n\tend \ell$, d'où le résultat.

\item
Considérons $u_n = (-1)^n$, qui n'admet pas de limite. Pourtant,
\[
  v_n = \frac{1}{n+1}\sum_{k=0}^n (-1)^k =
  \begin{cases}
    0 & \text{si $n$ est impair},\\[0.2em]
    \dfrac{1}{n+1} & \text{si $n$ est pair},
  \end{cases}
\]
donc $v_n\tend 0$. Ainsi, $(v_n)$ peut converger sans que $(u_n)$ ne converge.
\end{enumerate}
