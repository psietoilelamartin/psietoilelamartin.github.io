% Polynômes de Legendre


On note $E=\mathbb{R}_n[X]$, où $n \geqslant 1$.
\begin{enumerate}
\item Vérifier que:

$$
\langle P, Q\rangle=\int_{-1}^1 P(x) Q(x) \mathrm{d} x
$$

définit un produit scalaire sur $E$.
On note $\left(e_0, e_1, \ldots, e_n\right)$ la base obtenue par orthonormalisation de la base $\left(1, X, \ldots, X^n\right)$.
\item Pour tout entier $k \in\{1, \ldots, n\}$, on définit :

$$
f_k(X)=\frac{\mathrm{d}^k}{\mathrm{~d} X^k}\left(\left(X^2-1\right)^k\right)
$$

\begin{enumerate}
\item Déterminer le degré de $f_k$.
\item Calculer $\left\langle X^i, f_k\right\rangle$ pour $k \in\{1, \ldots, n\}$ et $i \in\{0, \ldots, k-1\}$.
\item En déduire que pour tout $k \in\{1, \ldots, n\}$, il existe un $\lambda_k$ tel que $f_k=\lambda_k e_k$.
\end{enumerate}
\end{enumerate}
