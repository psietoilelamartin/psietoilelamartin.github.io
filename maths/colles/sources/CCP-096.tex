% CCP 2023 exo 96
 Soit $X$ une variable aléatoire à valeurs dans $\mathbb{N}$, de loi de probabilité donnée par : $\forall n\in \mathbb{N}$, $P(X=n)=p_n$.\\
  La fonction génératrice de $X$ est notée $G_X$ et elle est définie par $G_X(t)=E[t^X]=\displaystyle \sum_{n=0}^{+\infty}p_nt^n$.
  \begin{enumerate}
  \item
  Prouver que l'intervalle $\left] -1,1\right[ $ est inclus dans l'ensemble de définition de $G_X$. \:\:
\item
Soit $X_1$ et $X_2$ deux variables aléatoires indépendantes à valeurs dans $\mathbb{N}$.\\
On pose $S=X_1+X_2$.\\
Démontrer  que  $\forall t\in \left]-1,1 \right[ $, $G_S(t)=G_{X_1} (t)G_{X_2}(t)$:\\
\begin{enumerate}
\item
en utilisant le produit de Cauchy de deux séries entières.  \:\:
\item
en utilisant uniquement la définition de la fonction génératrice par $G_X(t)=E[t^X]$. \:\:
\end{enumerate}

\textbf{Remarque}: on admettra, pour la question suivante,  que ce résultat est généralisable à $n$ variables aléatoires indépendantes à valeurs dans $\mathbb{N}$.
\item
Un sac contient quatre boules : une boule numérotée 0, deux boules numérotées 1 et une boule numérotée 2.\\
Soit $n\in\mathbb{N}^{*}$.
On effectue $n$ tirages  successifs, avec remise, d'une boule dans ce sac.\\
On note $S_n$ la somme des numéros tirés.\\
Soit $t\in \left]-1,1 \right[ $. \\
Déterminer $G_{S_n}(t)$ puis  en  déduire la loi de $S_n$.  \:\:
  \end{enumerate}
