\begin{enumerate}
\item Par comparaison à des séries de Riemann, $\zeta$ converge simplement sur 
$]1,+\infty[$.
\item S'il y avait convergence uniforme en $1$, alors 
$\displaystyle\sum\displaystyle\lim_{x\to1}\displaystyle\frac1{n^x}$ convergerait, 
ce qui n'est pas le cas.
\item Posons $f_n$ : $x\mapsto \displaystyle\frac1{n^x}$.\\
Les fonctions $f_n$ sont de classe $\mathscr{C}^{\infty}$ sur $] 1,+\infty[$ et
$$
f_n^{(k)}(x)=\displaystyle\frac{(-\ln n)^k}{n^x}.
$$
Sur $[a, b] \subset] 1,+\infty[$,
$$
\forall\, s \in[a, b],\ \left|f_n^{(k)}(x)\right| \leqslant \dfrac{(\ln n)^k}{n^a}.
$$
Soit $\rho \in]1, a[$, on a
$$
n^\rho \times \displaystyle\frac{(\ln n)^k}{n^a} 
\underset{n\to +\infty}\longrightarrow 0
$$
et il y a donc convergence de la série $\displaystyle\sum 
\displaystyle\frac{(\ln n)^k}{n^a}$.\\
Par majoration uniforme, la série de fonctions $\displaystyle\sum u_n^{(k)}$ converge 
normalement sur $[a, b]$.\\
Par convergence uniforme sur tout segment de $] 1,+\infty[$, 
on peut affirmer que $\zeta$ est de classe $\mathscr{C}^{\infty}$ sur $] 1,+\infty[$ et
$$
\zeta^{(p)}(x)=\displaystyle\sum_{n=1}^{+\infty} \displaystyle\frac{(-\ln n)^p}{n^x}.
$$
\item Monotonie :

$$
\zeta^{\prime}(x)=\displaystyle\sum_{n=1}^{+\infty} \displaystyle\frac{-\ln n}{n^x} 
\leqslant 0
$$
donc $\zeta$ est décroissante.\\
Convexité :

$$
\zeta^{''}(x)=\sum_{n=1}^{+\infty} \displaystyle\frac{(\ln n)^2}{n^x} \geqslant 0
$$
donc $\zeta$ est convexe.
\item Limite en $+\infty$ :

$$
\displaystyle\lim _{x \rightarrow+\infty} \displaystyle\frac{1}{n^x}= 
\begin{cases}0 & \text { si } n>1 \\ 1 & \text { si } n=1\end{cases}.
$$
Pour appliquer le théorème de la double limite, observons la convergence uniforme au
voisinage de $+\infty$.\\
Pour $x \geqslant 2$

$$
\left|f_n(x)\right| \leqslant \displaystyle\frac{1}{n^2}
$$


Or $\displaystyle\sum \displaystyle\frac{1}{n^2}$ converge, donc
$\displaystyle\sum u_n$ converge normalement et donc uniformément sur $[2,+\infty[$.
Par le théorème de la double limite

$$
\displaystyle\lim _{x \rightarrow+\infty} \zeta(x)=\displaystyle\sum_{n=1}^{+\infty} 
\displaystyle\lim _{x \rightarrow+\infty} \displaystyle\frac{1}{n^x}=1+0+0+\cdots=1.
$$

% Equivalent de $\zeta(s)-1$ quand $s \rightarrow+\infty$ :
% On a
%
% $$
% \zeta(s)-1=\frac{1}{2^x}+\sum_{n=3}^{+\infty} \frac{1}{n^x}
% $$
%
% avec
%
% $$
% 0 \leqslant \sum_{n=3}^{+\infty} \frac{1}{n^x} \leqslant \int_2^{+\infty} \frac{\mathrm{d} t}{t^x}=\frac{1}{(s-1)} \frac{1}{2^x}
% $$
%
% donc
%
% $$
% \zeta(s)-1=\frac{1}{2^x}+o\left(\frac{1}{2^x}\right) \sim \frac{1}{2^x}
% $$

\item
% Par monotonie, on peut affirmer que la fonction $\zeta$ admet une limite en $1^{+}$.
% Puisque
% $$
% \zeta(x) \geqslant \dsum_{k=1}^n \dfrac{1}{k^x}
% $$
% à la limite
% $$
% \dlim _{x \rightarrow 1^{+}} \zeta(x) \geqslant \dsum_{k=1}^n \dfrac{1}{k}
% $$
% Or ceci vaut pour tout $n$ et on sait que $\dsum_{k=1}^n \dfrac{1}{k} \tend+\infty$ donc
% $$
% \dlim _{x \rightarrow 1^{+}} \zeta(x)=+\infty.
% $$
La fonction $t \mapsto \displaystyle\frac{1}{t^x}$ est décroissante donc
$$
\displaystyle\int_n^{n+1} \displaystyle\frac{\mathrm{~d} t}{t^x} \leqslant 
\displaystyle\frac{1}{n^x} \leqslant \displaystyle\int_{n-1}^n 
\displaystyle\frac{\mathrm{~d} t}{t^x}
$$
On en déduit
$$
\displaystyle\int_1^{+\infty} \displaystyle\frac{\mathrm{d} t}{t^x} \leqslant \zeta(x) 
\leqslant 1+\displaystyle\int_1^{+\infty} \displaystyle\frac{\mathrm{d} t}{t^x}
$$
i.e.
$$
\displaystyle\frac{1}{x-1} \leqslant \zeta(x) \leqslant 1+\displaystyle\frac{1}{x-1}.
$$
Par suite
$$
\zeta(x) \underset{x \rightarrow 1}{\sim} \displaystyle\frac{1}{x-1}.
$$
\end{enumerate}
