Soit $u$ une application linéaire continue de $E$ dans $F$, deux espaces vectoriels normés non nuls. On définit :

$$
\begin{aligned}
& M_1=\sup\left\{\dfrac{\|u(x)\|}{\|x\|},\ x \in E \backslash\{0\}\right\} \\
& M_2=\sup\left\{\|u(x)\|,\ x \in E \text { t.q. }\|x\|=1\right\} \\
& M_3=\inf\left\{k \geqslant 0 \text { t.q. } \forall\, x \in E,\ \|u(x)\| \leqslant k\|x\|\right\}
\end{aligned}
$$

\begin{enumerate}
\item Justifier l'existence de ces nombres.
\item Montrer que $M_1=M_2=M_3$.
\end{enumerate}

\emph{Remarque :} On note en général $\left\vert\kern-0.25ex\left\vert\kern-0.25ex\left\vert u
    \right\vert\kern-0.25ex\right\vert\kern-0.25ex\right\vert$ ce nombre, et on peut montrer que $\left\vert\kern-0.25ex\left\vert\kern-0.25ex\left\vert .
    \right\vert\kern-0.25ex\right\vert\kern-0.25ex\right\vert$ définit sur $\mcal{L}(E, F)$ une norme. Cette norme s'appelle la \emph{\textbf{norme subordonnée}} à $\|\cdot\|_E$ et $\|\cdot\|_F$, et elle satisfait :
$$
\forall\, x \in E,\ \|u(x)\|_F \leqslant\left\vert\kern-0.25ex\left\vert\kern-0.25ex\left\vert u
    \right\vert\kern-0.25ex\right\vert\kern-0.25ex\right\vert\|x\|_E.
$$
