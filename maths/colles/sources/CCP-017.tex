% CCP 2023, ex. 17
Soit $A\subset \mathbb{C}$ et $\left( f_{n}\right)$ une suite de fonctions de $A$ dans $\mathbb{C}$.

\begin{enumerate}
\item D\'{e}montrer l'implication:
	\begin{eqnarray*}
	& \left( \text{la s\'{e}rie de fonctions }\displaystyle\sum f_{n}\ \text{converge uniform\'{e}ment sur $A$}\right)& \\
	&\Downarrow &\\
	&\left( \text{la suite de fonctions\ }\left( f_{n}\right) \ \text{converge uniform\'{e}ment vers 0 sur $A$}\right)&
	\end{eqnarray*}
\:\:\:\:
\item
On pose: $\forall\:n\in\mathbb{N}$, $\forall\:x\in\left[ 0;+\infty\right[ $, $f_n(x)=nx^2\mathrm{e}^{-x\sqrt{n}}$.\\
Prouver que $\displaystyle\sum f_n$ converge simplement sur $\left[ 0;+\infty\right[$.
\:\:\:\:\\
 $\displaystyle\sum f_n$ converge-t-elle uniformément sur $\left[ 0;+\infty\right[$? Justifier.
\:\:\:\:
\end{enumerate}
