\begin{enumerate}
\item Pour tout $\lambda\in\mathbb K$, $u$ et $\lambda\mathrm{Id}_E-v$ commutent, donc $u$ stabilise $\mathrm{Ker}(\lambda\mathrm{id}_E-v)$.
\item Puisque $u$ est diagonalisable, il admet un polynôme annulateur scindé à racines simples. Nécessairement, si $F$ est un sous-espace propre de $v$, ce même
polynôme annule aussi $u$\raisebox{-.5ex}{$|_F$}, qui est donc diagonalisable.
\item Pour chacun des sous-espaces propres $E_i$ de $v$ on choisit une base $\mathscr{B}_i$ dans laquelle l'endomrphisme enduit par $u$ admet une matrice diagonale. La concaténation de toutes ces bases est une base $\mathscr{B}$ de $E$ car $E$ est égal à la somme directe des sous-espaces propres de $v$.\\
Les vecteurs de toutes les $\mathscr{B}_i$ étant des vecteurs propres de $v$, la matrice de $v$ dans $\mathscr B$ est diagonale.\\
La matrice de $u$ dans $\mathscr{B}$ est diagonale par blocs, le bloc $i$ étant la matrice de $u$\raisebox{-.5ex}{$|_{E_i}$} dans $\mathscr B_i$. Tous ces blocs étant diagonaux par construction, la matrice de $u$ dans $\mathscr B$ est donc diagonale également.
\end{enumerate}
