% p.824, les oraux corrigés et commentés, Concours PC-PC*

% \section*{Thème : MARCHE ALÉatoire}
Une particule se déplace sur une surface comportant quatre positions successives, $A_{0}$ qui est un puits, $A_{1}$ et $A_{2}$ deux positions intermédiaires, $A_{3}$ un second puits.\\
A l'instant $t=n$,

\begin{itemize}
  \item si la particule est dans un puits, elle y reste
  \item si elle est en $A_{1}$, elle va en $A_{0}$ avec la probabilité $p$ et en $A_{2}$ avec la probabilité $1-p$
  \item si elle est en $A_{2}$, elle va en $A_{1}$ avec la probabilité $p$ et en $A_{3}$ avec la probabilité $1-p$.
\end{itemize}

On note $x_{n}$ la position de la particule à l'instant $t=n$, $x_{n}(\Omega)=\llbracket 0,3 \rrbracket$.

% \begin{enumerate}
%   \item Écrire en Python une fonction qui donne $x_{n+1}$ en fonction de $x_{n}$ et $p$.
%   \item Écrire en Python une fonction renvoyant $x_{n}$ en fonction de $x_{0}, n$ et $p$.
 % \item Faire l'histogramme des $x_{n}$ obtenues sur $N$ essais avec $p=1 / 2$.
% \end{enumerate}
On pose

$$
X_{n}=\left(\begin{array}{l}
P\left(x_{n}=0\right) \\
P\left(x_{n}=1\right) \\
P\left(x_{n}=2\right) \\
P\left(x_{n}=3\right)
\end{array}\right)
$$

\begin{enumerate}
\item Déterminer une matrice $M$ indépendante de $n$ telle que pour tout $n$, $X_{n+1}=M X_{n}$.\\
\item Montrer que $M$ est diagonalisable si et seulement si $0<p<1$.\\
\item On suppose $p=1 / 2$. %Diagonaliser $M$ %avec Python
% puis en déduire
Donner $\displaystyle\lim _{n \rightarrow \infty} X_{n}$. %Comparer aux résultats obtenus précédemment.
\item Si $X_0=\bpm \frac12\\0\\ \frac13\\ \frac16\epm$, calculer $\displaystyle\lim _{n \rightarrow \infty} X_{n}$.
\end{enumerate}
