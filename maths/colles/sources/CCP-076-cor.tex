% CCP 2023, ex. 76
\begin{enumerate}
\item
\begin{enumerate}
\item
Soit $E$ un $\mathbb{R}$-espace vectoriel muni d'un produit scalaire noté $\left(\:|\:\right)$.\\
On pose $\forall\:x\in E$, $||x||=\sqrt{(x|x)}$.

Inégalité de Cauchy-Schwarz: $\forall (x,y)\in E^2$, $|\left(x|y\right)|\leqslant ||x||\,||y||$\\
Preuve:\\
Soit $(x,y)\in E^2$.
Posons $\forall \lambda\in \mathbb{R}$, $P(\lambda)=||x+\lambda y||^2$.\\

On remarque que $\forall \lambda \in \mathbb{R}$, $P(\lambda )\geqslant 0$.\\
De plus,  $P(\lambda)=\left(x+\lambda y|x+\lambda y\right)$.\\
Donc,  par bilinéarité et symétrie de $\left(\:|\:\right)$, $P(\lambda )=||y||^2\lambda ^2+2\lambda \left(x|y\right)+||x||^2$.\\
On remarque que $P(\lambda)$ est un trinôme en $\lambda$ si et seulement si $||y||^2\neq 0$.\\
\textbf{Premier cas}: si $y=0$\\
Alors  $|\left(x|y\right)|=0$ et $||x||\,||y||=0$ donc l'inégalité de Cauchy-Schwarz est vérifiée.\\
\textbf{Deuxième cas}: $y\neq 0$\\
Alors $||y||=\sqrt{(y|y)}\neq 0$ car $y\neq 0$ et $\left(\:|\:\right)$ est une forme bilinéaire symétrique définie positive.\\
Donc, $P$ est un trinôme du second degré en $\lambda$ qui est positif ou nul.\\
On en déduit que le discriminant réduit $\Delta$ est négatif ou nul.\\
Or  $\Delta=\left(x|y\right)^2-||x||^2||y||^2$ donc  $\left(x|y\right)^2\leqslant||x||^2||y||^2$.\\
Et donc, $|\left(x|y\right)|\leqslant ||x||\,||y||$.
\item
On reprend les notations de 1. .\\
Prouvons que $\forall (x,y)\in E^2$, $|\left(x|y\right)|=||x||\,||y||$ $\Longleftrightarrow$ $x$ et $y$ sont colinéaires.\\
\medskip
Supposons que $|\left(x|y\right)|=||x||\,||y||$.\\
Premier cas: si $y=0$ \\
Alors $x$ et $y$ sont colinéaires.\\
Deuxième cas: si $y\neq 0$\\
Alors le discriminant de $P$ est nul et donc $P$ admet une racine double $\lambda_0$.\\
C'est-à-dire $P(\lambda_0)=0$ et comme  $\left(\:|\:\right)$ est définie positive, alors $x+\lambda_0y=0$.\\
Donc $x$ et $y$ sont colinéaires.\\
\medskip
Supposons que $x$ et $y$ soient  colinéaires.\\
Alors $\exists\:\alpha\in\mathbb{R}$ tel que $x=\alpha y$ ou $y=\alpha x$.\\
Supposons par exemple que $x=\alpha y$ (raisonnement similaire pour l'autre cas).\\
$|\left(x|y\right)|=|\alpha|.|\left(y|y\right)|=|\alpha|\,||y||^2$
et $||x||\,||y||=\sqrt{(x|x)}\,||y||=\sqrt{\alpha^2(y|y)}||y||=|\alpha|.||y||^2$.\\
Donc, on a bien l'égalité.\\
\medskip

\end{enumerate}
\item
On considère le produit scalaire classique sur $\mathcal{C}\left( \left[ a,b\right] ,\mathbb{R}\right)$ défini par :\\
$\forall (f,g)\in \mathcal{C}\left( \left[ a,b\right] ,\mathbb{R}\right)$, $(f|g)=\displaystyle\int_{a}^{b}f(t)g(t)dt$.\\
On pose $ \cE=\left\lbrace \displaystyle\int_{a}^{b}f(t)\mathrm{d}t\times \displaystyle\int_{a}^{b}\dfrac{1}{f(t)}\mathrm{d}t\:,\:f\in A\right\rbrace $.\\
$\cE\subset \mathbb{R}$.\\
$\cE\neq \emptyset$ car $(b-a)^2\in \cE$ ( valeur obtenue pour la fonction $t\longmapsto 1$ de $A$).\\
De plus, $\forall\:f\in A $,$\displaystyle\int_{a}^{b}f(t)\mathrm{d}t\times \displaystyle\int_{a}^{b}\dfrac{1}{f(t)}\mathrm{d}t\geqslant 0$  
donc $\cE$ est minorée par 0.\\
\medskip
On en déduit que $A$ admet une borne inférieure et on pose $m=\inf \cE$.\\
Soit $f\in A$.\\
On considère la quantité $\left( \displaystyle\int_{a}^{b}\sqrt{f(t)}\dfrac{1}{\sqrt{f(t)}}\mathrm{d}t\right) ^2$.\\
D'une part, $\left( \displaystyle\int_{a}^{b}\sqrt{f(t)}\dfrac{1}{\sqrt{f(t)}}\mathrm{d}t\right)  ^2=\left( \displaystyle\int_{a}^{b}1\mathrm{d}t\right) ^2=(b-a)^2.$\\
D'autre part, si on utilise l'inégalité de Cauchy-Schwarz pour le produit scalaire $(\:|\:)$  on obtient:\\
 $\left( \displaystyle\int_{a}^{b}\sqrt{f(t)}\dfrac{1}{\sqrt{f(t)}}\mathrm{d}t\right) ^2\leqslant \displaystyle\int_{a}^{b} f(t)\mathrm{d}t\displaystyle\int_{a}^{b}\dfrac{1}{f(t)}\mathrm{d}t$.\\
 \medskip
 On en déduit que $\forall\:f\in E$, $\displaystyle\int_{a}^{b} f(t)\mathrm{d}t\displaystyle\int_{a}^{b}\dfrac{1}{f(t)}\mathrm{d}t\geqslant (b-a)^2$.\\
 Donc $m\geqslant (b-a)^2$.\\
 Et, si on considère la fonction $f:t\longmapsto 1$ de $A$, alors $\displaystyle\int_{a}^{b} f(t)\mathrm{d}t\displaystyle\int_{a}^{b}\dfrac{1}{f(t)}\mathrm{d}t= (b-a)^2$.\\
 Donc $m=(b-a)^2$.

\end{enumerate}
