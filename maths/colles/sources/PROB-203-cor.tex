\begin{enumerate}
\item Par stabilité d'une tribu par réunion dénombrable, puis par intersection dénombrable, $B$ est bien un évènement.
\item $\omega\in B$ \ssi\ pour tout $n\in\N$, il existe $k\in\N$ tel que $k\geq n$ et $\omega\in A_k$. Ainsi $B$ est l'ensemble des élèments qui sont dans une infinité de $A_k$.
\item Posons, pour tout $p\in\N$, $U_p=\dbigcup_{k=p}^{\pinf} A_k$ et $B_p=\dbigcap_{n=0}^{p}\, \dbigcup_{k=n}^{\pinf} A_k=\dbigcap_{n=0}^{p}\, U_n$. On remarque que la suite $(U_p)_p$ est décroissante, et donc $B_p=U_p=\dbigcup_{n=p}^{+\infty} A_n$. La suite des $(B_p)_p$ est donc également décroissante, et par continuité décroissante
$$
P\left(B\right)=\dlim _{p \rightarrow+\infty}P\left(\dbigcup_{n=p}^{+\infty} A_n\right).
$$
Mais
$$
P\left(\bigcup_{n=p}^{+\infty} A_n\right) \leq \dsum_{n=p}^{+\infty} P\left(A_n\right)
$$
Et la suite des restes d'une série convergente converge vers 0 , ce qui permet de conclure.
\item
\begin{enumerate}
\item Dans ce cas, pour tout $p \in \N$ il existe $n_0\in\N$ tel que $n_0\geq p$ et $P(A_{n_0})=1$. Alors

$$
1\geq P\left(\bigcup_{n=p}^{+\infty} A_n\right)\geq P(A_{n_0})=1
$$

et le résultat cherché s'ensuit directement car l'égalité $P\left(B\right)=\dlim _{p \rightarrow+\infty}P\left(\dbigcup_{n=p}^{+\infty} A_n\right)$ est toujours valable.
\item Il faut avoir une première idée : l'indépendance se traduit mieux avec des intersections qu'avec des réunions d'événements. On va donc considérer l'évènement complémentaire de la limite supérieure :

$$
\barr B=\bigcup_{p=0}^{+\infty}\left(\bigcap_{n=p}^{+\infty} \overline{A_n}\right)
$$


Par continuité croissante,

$$
P(\barr B)=\lim _{p \rightarrow+\infty}\left(P\left(\bigcap_{n=p}^{+\infty} \overline{A_n}\right)\right)
$$

\item On a par croissance, et ensuite par indépendance des $\barr{A_n}$

$$
\forall q \geq p \quad P\left(\bigcap_{n=p}^{+\infty} \overline{A_n}\right) \leq P\left(\bigcap_{n=p}^q \overline{A_n}\right)=\prod_{n=p}^q\left(1-P\left(A_n\right)\right)
$$
\item
La deuxième idée est de prendre le logarithme, l'hypothèse portant sur une série. Pour cela nous utilisons
qu'au moins à partir du rang $p$, $P\left(A_n\right)<1$. Alors, si $p \leq q$,

$$
\ln \left(\prod_{n=p}^q\left(1-P\left(A_n\right)\right)\right)=\dsum_{n=p}^q \ln \left(1-P\left(A_n\right)\right)
$$


La suite $\left(P\left(A_n\right)\right)$ ne converge pas vers 0 a priori, donc on ne peut pas utiliser d'équivalents. On utilise donc une comparaison basée sur

$$
\forall\, x>-1, \quad \ln (1+x) \leq x
$$
Donc

$$
\ln \left(\dprod_{n=p}^q\left(1-P\left(A_n\right)\right)\right) \leq-\dsum_{n=p}^q P\left(A_n\right).
$$
\item Mais $\dsum P\left(A_n\right)=+\infty$, donc

$$
\dsum_{n=p}^q P\left(A_n\right) \xrightarrow[q \rightarrow+\infty]{}+\infty
$$

et donc

$$
\ln \left(\prod_{n=p}^q\left(1-P\left(A_n\right)\right)\right) \xrightarrow[q \rightarrow+\infty]{}-\infty
$$

et, donc,

$$
P\left(\bigcap_{n=p}^{+\infty} \overline{A_n}\right)=0
$$


Donc $P(\barr B)=0$, ce qui conclut.
\end{enumerate}
\end{enumerate}
