% CCP 2023 106
\begin{enumerate}
\item
$(U,V)(\Omega)=\left\lbrace (m,n)\in\mathbb{N}^2\:\text{tel que}\: m\geqslant n\right\rbrace$.
Soit $(m,n)\in\mathbb{N}^2$ tel que $m\geqslant n$.\\
\textbf{Premier cas: si m=n}\\
$P((U=m)\cap(V=n))=P((X=n)\cap(Y=n))=P(X=n)P(Y=n)$ car $X$ et $Y$ sont indépendantes.\\
Donc $P((U=m)\cap(V=n))=p^2q^{2n}$. \\
\textbf{Deuxième cas: si m>n}\\
$P((U=m)\cap(V=n))=P(\left[  (X=m)\cap(Y=n)\right]  \cup \left[  (X=n)\cap(Y=m)\right]  )$\\
Les événements $\left( (X=m)\cap(Y=n)\right)$ et  $\left( (X=n)\cap(Y=m)\right)$ sont incompatibles donc:\\
$P((U=m)\cap(V=n))=P\left( (X=m)\cap(Y=n)\right)+P\left( (X=n)\cap(Y=m)\right)$.\\
Or les variables $X$ et $Y$ suivent la même loi et sont indépendantes donc:\\
$P((U=m)\cap (V=n))=2P (X=m)P(Y=n)=2p^2q^{n+m}$.\\
\bigskip
\textbf{Bilan}: $P((U=m)\cap (V=n))=\left\lbrace
\begin{array}{ll}
p^2q^{2n}&\:\text{ si}\: m=n\\
2p^2q^{n+m} &\:\text{si}\: m>n\\
0 & \:\text{ sinon}
\end{array}
\right.$
\item
$U(\Omega)=\mathbb{N}$ et $V(\Omega)=\mathbb{N}.$
Soit $m\in\mathbb{N}$. \\
$P(U=m)=\displaystyle\sum\limits_{n=0}^{+\infty}P((U=m)\cap(V=n))$. ( loi marginale de $(U,V)$ )\\
Donc d'après 1., $P(U=m)=\displaystyle\sum\limits_{n=0}^{m}P((U=m)\cap(V=n))$ \:\:\: $(*)$\\
\textbf{ Premier cas}: $m\geqslant 1$\\
D'après $(*)$, $P(U=m)=P((U=m)\cap(V=m))+\displaystyle\sum\limits_{n=0}^{m-1}P((U=m)\cap(V=n))$. \\
Donc $P(U=m)=p^2q^{2m}+\displaystyle\sum\limits_{n=0}^{m-1}2p^2q^{n+m}
=p^2q^{2m}+2p^2q^m\displaystyle\sum\limits_{n=0}^{m-1}q^{n}
=p^2q^{2m}+2p^2q^m\dfrac{1-q^m}{1-q}=p^2q^{2m}+2pq^m(1-q^m)$\\
Donc  $P(U=m)=pq^m(pq^m+2-2q^m)$.\\
\medskip
\textbf{Deuxième cas }: $m=0$\\
D'après $(*)$ et 1., $P(U=0)=P((U=0)\cap (V=0))=p^2$.\\
\medskip
\textbf{Bilan }:
$\forall\:m\in\mathbb{N}$, $P(U=m)=pq^m(pq^m+2-2q^m)$.\\
\item
$W\left(\Omega \right)=\mathbb{N}^*$.\\
Soit $n\in\mathbb{N}^*$.\\
 $P(W=n)=P(V=n-1)=pq^{2(n-1)}(1+q)=(1-q)q^{2(n-1)}(1+q)$.\\
Donc $P(W=n)=(1-q^2)\left( q^2\right) ^{n-1}$.\\
Donc $W$ suit une loi géométrique de paramètre $1-q^2$.
% Donc, d'après le cours, $E(W)=\dfrac{1}{1-q^2}$.
% Donc $E(V)=E(W-1)=E(W)-1=\dfrac{q^2}{1-q^2}$.

\item
$P((U=0)\cap (V=1))=0$ et $P(U=0)P( V=1)=p^3q^2(1+q)\neq 0$.
Donc $U$ et $V$ ne sont pas indépendantes.

\end{enumerate}
