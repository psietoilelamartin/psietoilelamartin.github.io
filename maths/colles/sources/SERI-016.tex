On considère une suite $\left(u_n\right)_{n \geqslant 0}$ de nombres réels ou complexes. On définit la suite $\left(v_n\right)_{n \geqslant 0}$ par
$$
v_n=\dfrac{1}{n+1}\dsum_{k=0}^n u_k
$$
\begin{enumerate}
\item On suppose que la suite $\left(u_n\right)$ converge vers 0. Montrer que la suite $\left(v_n\right)$ converge vers 0 .\\
\emph{Indication :} soit $\varepsilon>0$. Montrer qu'il existe un rang $N$ tel que, si $n \geqslant N,\left|v_n\right| \leqslant \varepsilon$. Pour cela, couper $v_n$ en deux morceaux.
\item On suppose que la suite $\left(u_n\right)$ converge. Montrer que la suite $\left(v_n\right)$ converge, et a même limite que $(u_n)$. C'est le théorème de Césaro.
\item Montrer que la réciproque est fausse.
\end{enumerate}

On montrerait avec les mêmes outils que si $u_n\xrightarrow[n\to +\infty]{}+\infty$, $v_n$ aussi.
