Soit $f\ :\ \R\to \R$ une fonction continue et $(P_n )$ une suite de fonctions polynomiales convergeant
 uniformément vers $f$.
 \begin{enumerate}
\item Justifier qu'il existe un entier naturel $N$ tel que pour tout $n$ supérieur ou égal à $N$, on ait pour
 tout réel $x$, $|P_n (x ) - P_N (x ) |\leq 1$.\\
 Que peut-on en déduire quant au degré des fonctions polynômes $P_n - P_N$ lorsque $n\geq N$ ?
\item Conclure que $f$ est nécessairement une fonction polynomiale.
\end{enumerate}