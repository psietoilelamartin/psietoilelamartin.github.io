% ccp 2023, ex. 47
\begin{enumerate}
\item
On note $R$ le rayon de convergence de $\displaystyle\sum_{n\geqslant 1}^{} \dfrac{3^nx^{2n}}{n}$ et
pour tout réel $x$, on pose $u_n(x)=\dfrac{3^nx^{2n}}{n}$.\\
Pour $x$ non nul, $\left|\dfrac{u_{n+1}(x)}{u_n(x)}\right|=\left|\dfrac{3nx^2}{n+1}\right|\underset{n \to +\infty}{\longrightarrow}\left|3x^2\right|$.\\
Donc, d'après la règle de d'Alembert:\\
si $\left|3x^2\right|<1$ c'est-à-dire si $|x|<\dfrac{1}{\sqrt{3}}$ alors $\displaystyle\sum_{n\geqslant 1}^{} \dfrac{3^nx^{2n}}{n}$ converge absolument\\
et si $\left|3x^2\right|>1$ c'est-à-dire si $|x|>\dfrac{1}{\sqrt{3}}$ alors $\displaystyle\sum_{n\geqslant 1}^{} \dfrac{3^nx^{2n}}{n}$ diverge.\\
On en déduit que $R=\dfrac{1}{\sqrt{3}}$.\\
\medskip
On pose: $\forall\:x\in \left] -\dfrac{1}{\sqrt{3}},\dfrac{1}{\sqrt{3}}\right[$, $S(x)=\displaystyle\sum_{n= 1}^{+\infty} \dfrac{3^nx^{2n}}{n}$.\\
On a: $\forall\:x\in \left] -\dfrac{1}{\sqrt{3}},\dfrac{1}{\sqrt{3}}\right[$, $S(x)=\displaystyle\sum_{n= 1}^{+\infty} \dfrac{(3x^2)^n}{n}$.\\
Or, d'après les développements en séries entières usuels,
on a : $\forall\:t\in \left] -1,1\right[$, $\displaystyle\sum_{n= 1}^{+\infty} \dfrac{t^n}{n}=-\ln (1-t)$.\\
Ainsi :  $\forall\:x\in \left] -\dfrac{1}{\sqrt{3}},\dfrac{1}{\sqrt{3}}\right[$, $S(x)=-\ln (1-3x^2)$.\\

\item
Notons $R$ le rayon de convergence de $\displaystyle\sum_{n\geqslant 0} a_nx^n$.\\
On considère les séries $\displaystyle\sum_{n\geqslant 0} a_{2n}x^{2n}=\displaystyle\sum_{n\geqslant 0} 4^nx^{2n}$ et $\displaystyle\sum_{n\geqslant 0} a_{2n+1}x^{2n+1}=\displaystyle\sum_{n\geqslant 0} 5^{n+1}x^{2n+1}$.\\

Notons $R_1$ le rayon de convergence de $\displaystyle\sum_{n\geqslant 0} 4^nx^{2n}$ et $R_2$ le rayon de convergence de $\displaystyle\sum_{n\geqslant 0} 5^{n+1}x^{2n+1}$.\\
\medskip
Le rayon de convergence de $\displaystyle\sum_{n\geqslant 0} x^n$ vaut 1.\\
Or, $\displaystyle\sum_{n\geqslant 0} 4^nx^{2n}=\displaystyle\sum_{n\geqslant 0} (4x^2)^n$.\\
Donc pour $|4x^2|<1$ c'est-à-dire $|x|<\dfrac{1}{2}$,  $\displaystyle\sum_{n\geqslant 0} 4^nx^{2n}$ converge absolument\\
et pour $|4x^2|>1$ c'est-à-dire $|x|>\dfrac{1}{2}$,  $\displaystyle\sum_{n\geqslant 0} 4^nx^{2n}$ diverge.\\
On en déduit que $R_1=\dfrac{1}{2}$.\\
Par un raisonnement similaire et comme $\displaystyle\sum_{n\geqslant 0} 5^{n+1}x^{2n+1}=5x\displaystyle\sum_{n\geqslant 0} (5x^2)^n$, on trouve $R_2=\dfrac{1}{\sqrt{5}}$.\\
\medskip
$\displaystyle\sum_{n\geqslant 0} a_nx^n$ étant la série somme des séries $\displaystyle\sum_{n\geqslant 0} a_{2n}x^{2n}$ et $\displaystyle\sum_{n\geqslant 0} a_{2n+1}x^{2n+1}$, on en déduit, comme $R_1\neq R_2$,  que $R=\min(R_1,R_2)=\dfrac{1}{\sqrt{5}}$.\\
\medskip
D'après ce qui précéde, on en déduit également que:\\
 $\forall\:x\in \left] -\dfrac{1}{\sqrt{5}},\dfrac{1}{\sqrt{5}}\right[$, $S(x)=\displaystyle\sum_{n= 0}^{+\infty} a_nx^n=\displaystyle\sum_{n= 0}^{+\infty}(4x^2)^n+5x\sum_{n= 0}^{+\infty}(5x^2)^n=\dfrac{1}{1-4x^2}+\dfrac{5x}{1-5x^2}$.
 \end{enumerate}
