% CCP 2023 98
\begin{enumerate}
\item
L'expérience est la suivante:
l'épreuve de l'appel téléphonique de la secrétaire vers un correspondant est répétée $n$ fois et ces $n$ épreuves sont  mutuellement indépendantes.\\
De plus, chaque épreuve n'a que deux issues possibles:
 le correspondant est joint avec la probabilité $p$ (succès)
ou le correspondant n'est pas joint avec la probabilité $1-p$ (échec).\\
 La variable $X$ considérée représente le nombre de succès et suit donc une loi binômiale de paramètres $(n,p)$.\\
 C'est-à-dire $X(\Omega)=\llbracket 0,n\rrbracket $ et $\forall k\in{\llbracket 0,n\rrbracket}$ $P(X=k)=\dbinom{n}{k}p^{k}(1-p)^{n-k}$.\\

 \item
 \begin{enumerate}
 \item
 Soit $i$ $\in \llbracket 0,n \rrbracket $.\\
 Sous la condition $(X=i)$, la secrétaire rappelle $n-i$ correspondants lors de la seconde série d'appels.
 Comme cette nouvelle série d'appels se fait dans les mêmes conditions que pour la question 1.,  sauf pour le nombre de répétitions qui est maintenant égal à $n-i$ , alors on se retrouve dans le schéma d'une loi binomiale de paramètre $(n-i,p)$.\\
 \bigskip

Donc $P_{(X=i)}(Y=k)=\left\lbrace
\begin{array}{l}\dbinom{n-i}{k}p^{k}(1-p)^{n-i-k} \:\:\text{si}\:\: k\in \llbracket 0,n-i \rrbracket\\
0 \:\:\text{sinon}
\end{array}
\right.$
 \item
  $Z(\Omega)=\llbracket 0,n\rrbracket $ et $\forall k\in{\llbracket 0,n\rrbracket}$ $P(Z=k)=\displaystyle\sum\limits_{i=0}^{k}P(X=i\cap Y=k-i)=\displaystyle\sum\limits_{i=0}^{k}P_{(X=i)}(Y=k-i)P(X=i)$.\\

  Soit $k$ $\in \llbracket 0,n \rrbracket $. D'après les questions précédentes,\\ $P(Z=k)=\displaystyle\sum\limits_{i=0}^{k}\dbinom{n-i}{k-i}\dbinom{n}{i}p^{k}(1-p)^{2n-k-i}$.\\
  Or, d'après l'indication, $\dbinom{n-i}{k-i}\dbinom{n}{i}=\dbinom{k}{i}\dbinom{n}{k}$.
  Donc
  \begin{align*}
   P(Z=k)&=\displaystyle\sum\limits_{i=0}^{k}\dbinom{k}{i}\dbinom{n}{k}p^{k}(1-p)^{2n-k-i}\\&=\dbinom{n}{k}p^{k}(1-p)^{2n-k}\displaystyle\sum\limits_{i=0}^{k}\dbinom{k}{i}\left( \dfrac{1}{1-p}\right) ^i.
  \end{align*}

  Donc d'après le binôme de Newton,
  \begin{align*} P(Z=k)&=\dbinom{n}{k}p^{k}(1-p)^{2n-k}\left( \dfrac{2-p}{1-p}\right) ^{k}\\&=\dbinom{n}{k}\left( p(2-p)\right)^k\left( (1-p)^2\right)^{n-k}  .
  \end{align*}

  On vérifie que $1-p(2-p)=(1-p)^2$ et donc on peut conclure que:\\
   $Z$ suit une loi binomiale de paramètre $(n,p(2-p))$.\\
   \medskip
  \textbf{ Remarque}: preuve (non demandée dans l'exercice) de l'égalité proposée dans l'indication:\\
    $\dbinom{n-i}{k-i}\dbinom{n}{i}=\dfrac{(n-i)!}{(n-k)!\:(k-i)!}\dfrac{n!}{i!\:(n-i)!}=
  \dfrac{n!}{(k-i)!\:(n-k)!\:i!}=\dfrac{k!}{(k-i)!\:i!}\dfrac{n!}{k!\:(n-k)!}=\dbinom{k}{i}\dbinom{n}{k}$.\\
   \item
  D'après le cours, comme $Z$ suit une loi binomiale de paramètre $(n,p(2-p))$, alors:\\  $E(Z)=np(2-p)$ et $V(Z)=np(2-p)\left( 1-p(2-p)\right)=np(2-p)(p-1)^{2} $.


  \end{enumerate}
  \end{enumerate}
