Soit $f$ un endomorphisme d'un espace de dimension finie $n$ non nulle. On définit, pour tout entier naturel $p$ :
$$
F_p=\Ker\left(f^p\right) \quad \text { et } \quad G_p=\operatorname{Im}\left(f^p\right)
$$
( $f^p$ désigne l'itérée d'ordre $p$ de $f$ : $f^0=\mathrm{Id}$ et, $f^{p+1}=f \circ f^p$ ).
\begin{enumerate}
\item Démontrer que, des deux suites de s.e.v. $\left(F_p\right)$ et $\left(G_p\right)$, l'une est croissante et l'autre décroissante (pour l'inclusion).
\item Démontrer qu'il existe un plus petit entier naturel $r$ tel que $F_r=F_{r+1}$, et démontrer qu'alors, pour tout entier naturel $p$ supérieur ou égal à $r$, $F_p=F_{p+1}$.
\item Démontrer qu'il existe un plus petit entier naturel $s$ tel que $G_s=G_{s+1}$, et démontrer qu'alors, pour tout entier naturel $p$ supérieur ou égal à $s$, $G_p=G_{p+1}$. Y-a-t-il un lien entre $r$ et $s$ ?
\item Démontrer que $G_s$ et $F_r$ sont supplémentaires dans $E$.
% \item Soit $H_{k+1}$ un supplémentaire dans $F_{k+2}$ de $F_{k+1}$. Démontrer que la restriction de $f$ à $H_{k+1}$ est injective, que $f\left(H_{k+1}\right)$ est un sous-espace vectoriel de $F_{k+1}$ et qu'il est en somme directe avec $F_k$. En déduire que la suite $\left(\alpha_k\right)$, où $\alpha_k=\operatorname{dim} F_{k+1}-\operatorname{dim} F_k$, est décroissante.
\end{enumerate}
