\begin{enumerate}
\item
\begin{enumerate}
\item Dans $\mathscr{M}_n(\mathbb{C})$, $A$ est trigonalisable et lors de cette trigonalisation, les valeurs propres de $A$ apparaissent sur la diagonale. Donc $A$ est semblable à une matrie triangulaire supérieure avec des complexes $\lambda_1,\ldots,\lambda_n$ sur la diagonale. Alors pour tout $k\in\mathbb{N}$, la diagonale de $A^k$ vaut $\lambda_1^k,\ldots,\lambda_n^k$. Si l'un des $\lambda_j\neq0$, alors cette diagonale n'est jamais nulle, ce qui est contradictoire avec la nilpotence de $A$. Donc $A$ est bien semblable à une matrice triangulaire strictement supérieure. Et on remarque alors que le polynôme caractéristique de $A$ vaut $X^n$.
\item Pour $A \in \mcal{M}_n(\mathbb{R})$, on a aussi $A \in \mcal{M}_n(\mathbb{C})$ et le polynôme caractéristique est calculé par la même formule dans les deux cas. Par suite le polynôme caractéristique pour $A \in \mcal{M}_n(\mathbb{R})$ est scindé et donc à nouveau $A$ est trigonalisable avec des 0 sur la diagonale.
\end{enumerate}
\item
Il existe une base dans laquelle la matrice $M$ de $u$ est triangulaire supérieure avec les valeurs propres de $u$, $(\lambda_1,\ldots,\lambda_n)$, sur la digonale. Alors par récurrence et propriété du produit des matrices triangulaires, pour tout $k\in\mathbb{N}$ les coefficients diagonaux de $M^k$ sont les $(\lambda_1^k,\ldots,\lambda_n^k)$. Et ensuite, par linéarité, si $P\in\mathbb{C}[X]$, les coefficients diagonaux de $P(M)$ sont les $(P(\lambda_1),\ldots,P(\lambda_n))$. Ainsi le spectre de $P(u)$ est $\left\{P(\lambda_1),\ldots,P(\lambda_n)\right\}$, c'est-à-dire $P\p{\Sp(u)}$.                                                                                                                                                                                                                                                                                                                                                                                                                                                                                                 \end{enumerate}
