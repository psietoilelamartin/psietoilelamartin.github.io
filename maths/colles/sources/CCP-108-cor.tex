% CCP 2023 exo 108
\begin{enumerate}
\item
 $\forall\:(i,j)\in\mathbb{N}^2$, $P((X=i)\cap (Y=j))=\dfrac{1}{\mathrm{e}\:2^{i+1}j!}$.\\
 $X(\Omega)=\mathbb{N}$. \\
Soit $i\in \mathbb{N}$.\\
$\displaystyle\sum\limits_{j\geqslant 0}^{}\dfrac{1}{\mathrm{e}\:2^{i+1}j!}=\dfrac{1}{\mathrm{e}\:2^{i+1}}\displaystyle\sum\limits_{j\geqslant 0}^{}\dfrac{1}{j!}$ converge et $\displaystyle\sum\limits_{j=0}^{+\infty}\dfrac{1}{\mathrm{e}\:2^{i+1}j!}=\dfrac{1}{2^{i+1}}$.\\
Or $P(X=i)=\displaystyle\sum\limits_{j=0}^{+\infty}P((X=i)\cap (Y=j))$
donc $P(X=i)=\displaystyle\sum\limits_{j=0}^{+\infty}\dfrac{1}{\mathrm{e}2^{i+1}j!}
=\dfrac{1}{\mathrm{e}2^{i+1}}\displaystyle\sum\limits_{j=0}^{+\infty}\dfrac{1}{j!}
=\dfrac{1}{2^{i+1}}.$\\
Conclusion: $\forall\:i\in\mathbb{N}$, $P(X=i)=\dfrac{1}{2^{i+1}}.$\\
\bigskip
 $Y(\Omega)=\mathbb{N}$.\\
 Soit $j\in \mathbb{N}$.\\
  $\displaystyle\sum\limits_{i\geqslant 0}^{}\dfrac{1}{\mathrm{e}\:2^{i+1}j!}=
  \dfrac{1}{2\mathrm{e}j!}\displaystyle\sum\limits_{i\geqslant 0}^{}\left(\dfrac{1}{2} \right)^i$ converge (série géométrique de raison $\dfrac{1}{2}$)   et $\displaystyle\sum\limits_{i=0}^{+\infty}\dfrac{1}{\mathrm{e}\:2^{i+1}j!}=\dfrac{1}{2\mathrm{e}j!}\dfrac{1}{1-\dfrac{1}{2}}=\dfrac{1}{\mathrm{e}j!}$.\\
Or $P(Y=j)=\displaystyle\sum\limits_{i=0}^{+\infty}P((X=i)\cap (Y=j))$.\\
Donc $P(Y=j)=\displaystyle\sum\limits_{i=0}^{+\infty}\dfrac{1}{\text{e}2^{i+1}j!}
=\dfrac{1}{2\mathrm{e}j!}\displaystyle\sum\limits_{i=0}^{+\infty}\left(\dfrac{1}{2} \right)^i
=\dfrac{1}{2\mathrm{e}j!}\dfrac{1}{1-\dfrac{1}{2}}=\dfrac{1}{\mathrm{e}j!}$.\\
Conclusion: $\forall\:j\in\mathbb{N}$, $P(Y=j)=\dfrac{1}{\mathrm{e}j!}$.
\item
\begin{enumerate}
\item
On pose $Z=X+1$.\\
$Z(\Omega)=N^*$.\\
De plus, $\forall\:n\in\mathbb{N}^*$, $P(Z=n)=P(X=n-1)=\dfrac{1}{2^n}=\dfrac{1}{2}\left( \dfrac{1}{2}\right)^{n-1}$. \\
Donc $Z$ suit une loi géométrique de paramètre $p=\dfrac{1}{2}$.\\
\medskip
Donc, d'après le cours, $E(Z)=\dfrac{1}{p}=2$ et $V(Z)=\dfrac{1-p}{p^2}=2$.\\
Donc $E(X)=E(Z-1)=E(Z)-1=2-1=1$ et $V(X)=V(Z-1)=V(Z)=2$.\\
C'est-à-dire $E(X)=1$ et $V(X)=2$.\\
\item
$Y$ suit une loi de Poisson de paramètre $\lambda=1$.\\
Donc, d'après le cours, $E(Y)=V(Y)=\lambda=1$.
\end{enumerate}
\item
On a : $\forall\:(i,j)\in\mathbb{N}^2$, $P((X=i)\cap(Y=j))=P(X=i)P(Y=j)$.
Donc les variables $X$ et $Y$ sont indépendantes.
\item $\left((X=k)\right)_{k\in\N}$ est un système complet d'évènements donc avec la formule des probabilités totales
% $(X=Y)=\displaystyle\bigcup\limits_{k\in\mathbb{N}}^{}((X=k)\cap (Y=k))$ et il s'agit d'une union d'événements deux à deux incompatibles donc~:
\begin{align*}
P(X=Y)&=\displaystyle\sum\limits_{k=0}^{+\infty}P((X=k)\cap (Y=k))
=\displaystyle\sum\limits_{k=0}^{+\infty}\dfrac{1}{\text{e}2^{k+1}}\dfrac{1}{k!}\\
&=\dfrac{1}{2\mathrm{e}}\displaystyle\sum\limits_{k=0}^{+\infty}\dfrac{\left( \dfrac{1}{2}\right) ^k}{k!}
=\dfrac{1}{2\mathrm{e}}\mathrm{e}^{\frac{1}{2}}.
\end{align*}
Donc $P(X=Y)=\dfrac{1}{2\sqrt{\mathrm{e}}}$.

\end{enumerate}
