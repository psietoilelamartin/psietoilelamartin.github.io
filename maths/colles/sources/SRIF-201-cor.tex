Commençons par observer que
\begin{align*}
\displaystyle\frac{\sin t}{\mathrm{e}^t-1}&=\sin t\times\displaystyle\frac{\e^{-t}}{1-\e^{-t}}\\
&=\displaystyle\sum_{n=1}^{+\infty} \sin t \times \mathrm{e}^{-n t}.
\end{align*}
De plus $t \mapsto \sin t \times \mathrm{e}^{-n t}$ est intégrable sur $] 0,+\infty[$ par comparaison à une série exponentielle,
et
\begin{align*}
\displaystyle\int_0^{+\infty}|\sin t| \mathrm{e}^{-n t} \mathrm{~d} t &\leqslant \displaystyle\int_0^{+\infty} t \mathrm{e}^{-n t} \mathrm{~d} t\\
&\leq\displaystyle\frac{1}{n^2}
 \end{align*}
et ce dernier terme est le terme général d'une série convergente,
donc $t \mapsto \displaystyle\frac{\sin t}{\mathrm{e}^t-1}$ est intégrable sur $]0,+\infty[$.\\
Enfin, nous pouvons utiliser le théorème d'intégration terme à terme sur un intervalle quelconque, et ainsi
\begin{align*}
\displaystyle\int_0^{+\infty} \displaystyle\frac{\sin t}{\mathrm{e}^t-1} \mathrm{~d} t&=\displaystyle\sum_{n=1}^{+\infty} \displaystyle\int_0^{+\infty} \sin t . \mathrm{e}^{-n t} \mathrm{~d} t.\\
\text{Or }\ \displaystyle\int_0^{+\infty} \sin t . \mathrm{e}^{-n t} \mathrm{~d} t&=\operatorname{Im} \int_0^{+\infty} \mathrm{e}^{(-n+i) t} \mathrm{~d} t\\&
=\displaystyle\frac{1}{n^2+1}.
 \end{align*}
Finalement
$$
\int_0^{+\infty} \frac{\sin t}{\mathrm{e}^t-1} \mathrm{~d} t=\sum_{n=1}^{+\infty} \frac{1}{n^2+1}.
$$
