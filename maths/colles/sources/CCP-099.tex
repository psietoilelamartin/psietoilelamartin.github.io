% CCP 2023 exo 99
\begin{enumerate}
\item
Rappeler l'inégalité de Bienaymé-Tchebychev.\:\:\:\:
\item
 Soit $(Y_n)$ une suite de variables aléatoires  indépendantes, de même loi et et telle que $\forall n\in \mathbb{N}$, $Y_n$ admet une variance.\\ On pose $S_n=\displaystyle\sum\limits_{k=1}^{n}Y_k$.\\

Prouver que: $\forall\:a\in \left] 0,+\infty\right[ $, $P\left( \left|\dfrac{S_n}{n}-E(Y_1)\right|\geqslant a\right) \leqslant\dfrac{V(Y_1)}{na^2}$.\:\:\:\:
\item \textbf{Application}:
On effectue des tirages successifs, avec remise, d'une boule dans une urne contenant 2 boules rouges et 3 boules noires.\\
\`A partir de quel nombre de tirages peut-on garantir à plus de 95\% que la proportion de boules rouges obtenues restera comprise entre $0,35$ et $0,45$? \:\:\:\:\\
\textbf{Indication} : considérer la suite $(Y_i)$ de variables aléatoires de Bernoulli où $Y_i$ mesure l'issue du $i^{\text{ème}}$  tirage.

\end{enumerate}
