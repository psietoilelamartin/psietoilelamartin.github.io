\begin{align*}
\begin{vmatrix}X&-3&-2\\2&X-5&-2\\-2&3&X\end{vmatrix} &\underset{\substack{L_1\leftarrow L_1+L_3 \\ L_2\leftarrow L_2+L_3}}{=} \begin{vmatrix}X-2&0&X-2\\0&X-2&X-2\\-2&3&X\end{vmatrix}\\
&=(X-2)^2\begin{vmatrix}1&0&1\\0&1&1\\-2&3&X\end{vmatrix}\\
&=(X-2)^2\begin{vmatrix}1&0&0\\0&1&1\\-2&3&X+2\end{vmatrix}\\
&=(X-2)^2\begin{vmatrix}1&1\\3&X+2\end{vmatrix}\\
&=(X-2)^2(X-1).
\end{align*}

Ensuite $A-2\mathrm{I}_3=\begin{pmatrix} -2&3&2\\-2&3&2\\2&-3&-2\end{pmatrix}$, dont on observe que le noyau contient $\begin{pmatrix} 1\\0\\1\end{pmatrix}$ et $\begin{pmatrix} 3\\2\\0\end{pmatrix}$, qui en constituent une base.\\
Et $A-\mathrm{I}_3=\begin{pmatrix} -&3&2\\-2&4&2\\2&-3&-1\end{pmatrix}$ ,dont on observe que le noyau est engendré par $\begin{pmatrix} 1\\1\\-1\end{pmatrix}$.\\
Ainsi $A=PDP^{-1}$ avec $P=\begin{pmatrix} 1&3&1\\0&2&1\\1&0&-1\end{pmatrix}$ et $D=\begin{pmatrix} 2&0&0\\0&2&0\\0&0&1\end{pmatrix}$.
