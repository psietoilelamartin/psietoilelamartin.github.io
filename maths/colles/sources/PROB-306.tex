% p.739, les oraux corrigés et commentés, Concours PC-PC*


% THÈME : SUITE ARITHMÉTICO-GÉOMÉTRIQUE\\
On considère deux urnes $U$ et $V$. L'urne $U$ contient deux boules blanches et quatre boules noires. L'urne $V$ contient trois boules blanches et trois boules noires.\\
On effectue des tirages successifs avec remise selon la procédure suivante. Pour le premier tirage, on choisit une urne au hasard et on tire une boule dans l'urne choisie. On note sa couleur et on la remet dans l'urne.

\begin{itemize}
  \item Si la boule tirée était blanche, le tirage suivant s'effectue dans l'urne $U$.
  \item Sinon, le tirage suivant s'effectue dans l'urne $V$.
\end{itemize}

On itère le procédé pour les tirages suivants. Pour tout entier $n \in \mathbb{N}^{*}$, on note $B_{n}$ l'evénement «la boule tirée au $n^{\mathrm{e}}$ tirage est blanche» et $p_{n}=P\left(B_{n}\right)$.

\begin{enumerate}
  \item Calculer $p_{1}$.
  \item Déterminer $p_{n+1}$ en fonction de $p_{n}$.
  \item Déterminer la limite de la suite $\left(p_{n}\right)$.
\end{enumerate}
