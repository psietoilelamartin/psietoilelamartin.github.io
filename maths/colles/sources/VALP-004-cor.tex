Soit $\lambda \in \mathbb{K}$ et $P \in \mathbb{K}[X]$.
$$
\varphi(P)=\lambda P \Leftrightarrow X P^{\prime}(X)=\lambda P(X).
$$

\bu\ Analyse :
Si cette équation possède une solution $P \neq 0$ alors en posant $n=\operatorname{deg} P$, on peut écrire $P=a_n X^n+\cdots+a_1 X+a_0$ avec $a_n \neq 0$. L'équation $X P^{\prime}(X)=\lambda P(X)$ donne
$$
\forall\,k\in\llbr 0,n\rrbr,\ \lambda a_k=n a_k.
$$

Sachant $a_n \neq 0$, on obtient $\lambda=n$ et $a_{n-1}=\ldots=a_1=a_0=0$.
Ainsi
$$
\lambda \in \mathbb{N} \text { et } P=a_\lambda X^\lambda
$$

\bu\ Synthèse : Soit $\l\in\N$. On remarque tout de suite que $\ffi(X^n)=nX^n$, donc $\l$ est bien une \valp\ de $\varphi$.\\

\bu\ Conclusion : l'ensemble des \valps\ de $\ffi$ est \N.
