On considère pour tout $n\in\mathbb N^\ast$ la suite $u_n=\displaystyle\sum_{k=1}^n \dfrac1k -\ln n$.\\
L'objectif est de montrer que $(u_n)$ converge. Sa limite est appelée \textbf{\emph{constante d'Euler}} et notée $\gamma$.\\
Nous allons employer deux méthodes.

\subsubsection{Comparaison série-intégrale et série télescopique}

Pour tout $n\in\N^\ast$, on note $H_n=\displaystyle\sum_{k=1}^n\dfrac1k$.
\begin{enumerate}
 \item Montrer que $H_n\sim \ln n$.
 \item Montrer que $u_{n+1}-u_n=O\p{\dfrac1{n^2}}$.
 \item On pose pour tout $n>0$, $v_n=u_{n+1}-u_n$. Donner la nature de $\displaystyle\sum v_n$ et conclure.
\end{enumerate}

\subsubsection{Deux suites adjacentes}

Pour tout $n\in\N^\ast$ on pose $w_n=u_n+\ln n-\ln(n+1)$. Montrer que $(u_n)$ et $(w_n)$ sont adjacentes et conclure.
