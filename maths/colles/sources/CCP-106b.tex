% CCP 2023 exo 104
$X$ et $Y$ sont deux variables aléatoires indépendantes et à valeurs dans $\mathbb{N}$.\\
Elles suivent la même loi définie par:
$\forall\:k\in\mathbb{N}$, $P(X=k)=P(Y=k)=pq^k$ où    $p \in \left] 0,1\right[ $ et $q=1-p$.\\
On considère alors les variables $U$ et $V$ définies par $U=\max(X,Y)$ et $V=\min(X,Y)$.
\begin{enumerate}
\item
Déterminer la loi du couple $(U,V)$.
\item
Déterminer la loi marginale de $U$.\\
On admet que $V(\Omega)=\mathbb{N}$ et que, $\forall\,n\in\mathbb{N}$, $P(V=n)=pq^{2n}(1+q)$.
\item
Prouver que $W=V+1$ suit une loi géométrique.
% En déduire l'espérance de $V$.
\item
$U$ et $V$ sont-elles indépendantes?

\end{enumerate}
