\textbf{Analyse :} soit $f$ une telle fonction. Soit $i,j,k\in\intn$. \\
Notons $\l=f(E_{1,1})$. Puisque $E_{i,1}E_{1,i}=E_{i,i}$ et $E_{1,i}E_{i,1}=E_{1,1}$, il vient $f(E_{i,i})=\l=\l \tr E_{i,i}$.
Si $i\neq j$, alors $E_{i,k}E_{k,j}=E_{i,j}$ et $E_{k,j}E_{i,k}=0$, donc $f(E_{i,j})=0=\l \tr E_{i,j}$.\\
Finalement, pour tout $i,j\in\intn$, $f(E_{i,j})=\l \tr E_{i,j}$. Puisque les $E_{i,j}$ forment une base de $\mcal M_n(\mathbb K)$, il vient $f=\l\tr$.\\
\textbf{Synthèse :} on sait que pour tout $\l\in\K$, $\l \tr$ vérifie la propriété étudiée.\\
\textbf{Conclusion :} l'ensemble des formes linéaires $f$ sur $\mcal M_n(\mathbb K)$ vérifiant $\forall\,A,B \in \mcal M_n(\K),~ f(AB) = f(BA)$ est
$$\ens{\l \tr,\ \l\in\K}.$$
