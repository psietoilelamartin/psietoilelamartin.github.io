 On pose, pour tout entier naturel $n$, $I_n=\displaystyle\int_0^{\pi /2}(\sin
x)^n\dx$.
\begin{enumerate}
\item Calculer $I_0$ et $I_1$.\\
Pour tout $n\geq2$, donner une relation de récurrence entre $I_n$
et $I_{n-2}$.\\
En déduire, pour tout $n\in\N$, la valeur de $I_n$ selon la parité de $n$.
\item Montrer que la suite $(I_n)$ est décroissante. En déduire
$\lim\limits_{n\to+\infty }\dfrac{I_{n-1}}{I_n}$.
\item Montrer : $\forall n\geqslant 1,\quad nI_nI_{n-1}=\dfrac{\pi }{2}$. En
déduire $\lim\limits_{n\to+\infty }I_n$ et $\lim\limits_{n\to+\infty }I_n\sqrt{n}$.
\item Montrer que : $\lim\limits_{n\to+\infty }2n(I_{2n})^2=\dfrac{\pi }{2}$.\\ En
déduire que : $\lim\limits_{n\to+\infty
}\left[n\left(\dfrac{1.3.5.\dots(2n-1)}{2.4.6\dots
2n}\right)^2\right]=\dfrac{1}{\pi }\qquad$ (\textbf{formule de Wallis}).
\end{enumerate}



