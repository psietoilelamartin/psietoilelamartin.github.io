% Monnier PC-PSI-PT
% 5.31 Recherche d'équivalents d'intégrales à paramètre entier naturel\\
D'abord, pour tout $n\in\N^\ast$ l'intégrale

$I_{n}=\displaystyle\int_{0}^{1} \ln \left(1+x^{n}\right) \mathrm{d} x$, existe comme intégrale d'une application continue sur un segment.

Effectuons un changement de variable. Posons $\varphi(x)=x^n$.
Alors
\begin{align*}
I_n&=\displaystyle\int_0^1\ln(1+x^n)\mathrm{d} x =\dfrac 1n \displaystyle\int_0^1 (nx^{n-1})\times \dfrac x{x^n}\ln(1+x^n)\mathrm{d} x\\
&= \dfrac 1n\displaystyle\int_0^1 \varphi'(x)\dfrac{(\varphi(x))^{1/n}}{\varphi(x)}\ln(1+\varphi(x))\mathrm{d} x\\
&= \dfrac 1n\displaystyle\int_{\varphi(0)}^{\varphi(1)} \dfrac{t^{1/n}}{t}\ln(1+t)\dt\\
&= \dfrac{1}{n} \underbrace{\displaystyle\int_{0}^{1} t^{\frac{1}{n}} \dfrac{\ln (1+t)}{t} \mathrm{~d} t}_{\text {notée } J_{n}}
\end{align*}

où $J_{n}$ est d'ailleurs une intégrale de fonction intégrable sur $]0, 1]$.

Pour obtenir la limite de $J_{n}$ (si elle existe), nous allons utiliser le théorème de convergence dominée.

Notons, pour tout $n \in \mathbb{N}^{*}$ :

$$
f_{n}\ :\ ] 0,1] \longrightarrow \mathbb{R},\ t \longmapsto t^{\frac{1}{n}} \dfrac{\ln (1+t)}{t}
$$

\begin{itemize}
  \item Pour tout $n \in \mathbb{N}^{*}, f_{n}$ est continue par morceaux (car continue) sur $] 0,1]$.

  \item $f_{n} \underset{n\to\pinf}{\stackrel{C . S}{\longrightarrow}} f$, où $f$ : $] 0,1] \longrightarrow \mathbb{R}$, $t \longmapsto \dfrac{\ln (1+t)}{t}$, car, pour $t \in] 0,1]$ fixé, on a $t^{\frac{1}{n}} \underset{n\to\pinf}{\longrightarrow} 1$.

  \item $f$ est continue par morceaux (car continue) sur $] 0,1]$.

  \item On a, pour tout $n \in \mathbb{N}^{*}$ et tout $t \in] 0,1]$ :

\end{itemize}

$$
\left|f_{n}(t)\right|=t^{\frac{1}{n}} \dfrac{\ln (1+t)}{t} \leqslant \dfrac{\ln (1+t)}{t}
$$

et l'application $t \longmapsto \dfrac{\ln (1+t)}{t}$ est continue par morceaux (car continue), positive, intégrable sur $] 0,1]$, puisque $\dfrac{\ln (1+t)}{t} \underset{t \rightarrow 0}{\longrightarrow} 1$. Ceci montre que la suite $\left(f_{n}\right)_{n \geqslant 1}$ vérifie l'hypothèse de domination.

D'après le théorème de convergence dominée :

$$
\displaystyle\int_{0}^{+\infty} f_{n} \underset{n\to\pinf}{\longrightarrow} \displaystyle\int_{0}^{+\infty} f
$$

Ainsi : $\quad J_{n} \underset{n\to\pinf}{\longrightarrow} \displaystyle\int_{0}^{+\infty} \dfrac{\ln (1+t)}{t} \mathrm{~d} t=\dfrac{\pi^{2}}{12}$.

On conclut : $\quad \displaystyle\int_{0}^{+\infty} \ln \left(1+x^{n}\right) \mathrm{d} x \underset{n\to\pinf}{\sim} \dfrac{\pi^{2}}{12} \dfrac{1}{n}$.
