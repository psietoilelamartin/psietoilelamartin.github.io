% CCP 2023 exo 98
Une secrétaire effectue, une première fois, un appel téléphonique vers $n$ correspondants distincts.\\
On admet que les $n$ appels constituent $n$ expériences indépendantes et que, pour chaque appel, la probabilité d'obtenir le correspondant demandé est $p$ ($p\in{\left]  0,1\right[ }$).\\
Soit $X$ la variable aléatoire représentant le nombre de correspondants obtenus.
\begin{enumerate}
\item Donner la loi de $X$. Justifier.\:\:\:\:
\item
La secrétaire rappelle une seconde fois, dans les mêmes conditions, chacun des $n-X$ correspondants qu'elle n'a pas pu joindre au cours de la première série d'appels.
On note $Y$ la variable aléatoire représentant le nombre de personnes jointes au cours de la seconde série d'appels.
\begin{enumerate}
\item
Soit $i\in \llbracket 0,n \rrbracket $.
Déterminer, pour  $k\in \mathbb{N}, $ $P_{(X=i)}(Y=k)$.\:\:\:\:
\item
Prouver que $Z=X+Y$ suit une loi binomiale dont on déterminera le paramètre.\:\:\:\:\\
\textbf{Indication} : on pourra utiliser, sans la prouver, l'égalité suivante: $\dbinom{n-i}{k-i}\dbinom{n}{i}=\dbinom{k}{i}\dbinom{n}{k}$.\\
\item
Déterminer l'espérance et la variance de $Z$.\:\:\:\:
\end{enumerate}
\end{enumerate}
