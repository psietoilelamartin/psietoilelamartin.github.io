\begin{enumerate}
\item Soit $f\in A$. Alors $\left|f(0)-g(0)\right|=1$ donc $\|f-g\|_\infty\geqslant 1$. Donc $\mathscr B\left(g,\frac12\right)\cap A=\emptyset$ : $g$ n'est pas adhérent à $A$ pour $\| .\|_\infty$.
\item Soit $f_n$ telle que $f_n(x)=1$ si $x>\displaystyle\frac1n$ et $f(x)=nx$ si $x\in\left[0,\frac1n\right]$. Alors $f_n\in A$ et $\|g-f_n\|_1=\displaystyle\int_0^{1/n}(1-nx)\mathrm d x=\displaystyle\frac1{2n}$. Donc $g$ est limite d'une suite d'éléments de $A$ : c'est un point adhérent à $A$ pour $\|.\|_1$.
\end{enumerate}
